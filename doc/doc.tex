\documentclass[11pt,a4paper]{article}

\usepackage{english}
\usepackage[utf8]{inputenc}
\usepackage{listings}
\title{x86 string test}
\author{Michał Sypetkowski}
\date{}

\begin{document}
\maketitle
\section{Introduction}\label{sec:intro}
Various implementations of basic 4 string algorithms (memcpy, strcmp, strlen and strstr) were measured by terms of performance.
Standard C library implementations of these algorithms were also included in comparison.

\subsection{environment}\label{subsec:environment}
\begin{lstlisting}[]
g++ --version
g++ (GCC) 7.1.1 20170528
Copyright (C) 2017 Free Software Foundation, Inc.
This is free software; see the source for copying conditions.  There is NO warranty;
not even for MERCHANTABILITY or
FITNESS FOR A PARTICULAR PURPOSE.


uname -a
Linux qwe 4.11.3-1-ARCH #1 SMP PREEMPT
    Sun May 28 10:40:17 CEST 2017 x86_64 GNU/Linux


lscpu
Architecture:          x86_64
CPU op-mode(s):        32-bit, 64-bit
Byte Order:            Little Endian
CPU(s):                8
On-line CPU(s) list:   0-7
Thread(s) per core:    2
Core(s) per socket:    4
Socket(s):             1
NUMA node(s):          1
Vendor ID:             GenuineIntel
CPU family:            6
Model:                 60
Model name:            Intel(R) Core(TM) i7-4700MQ CPU @ 2.40GHz
Stepping:              3
CPU MHz:               2400.146
CPU max MHz:           3400.0000
CPU min MHz:           800.0000
BogoMIPS:              4790.12
Virtualization:        VT-x
L1d cache:             32K
L1i cache:             32K
L2 cache:              256K
L3 cache:              6144K
NUMA node0 CPU(s):     0-7
Flags:                 fpu vme de pse tsc msr pae mce cx8
apic sep mtrr pge mca cmov pat pse36 clflush dts acpi mmx
fxsr sse sse2 ss ht tm pbe syscall nx pdpe1gb rdtscp lm
constant_tsc arch_perfmon pebs bts rep_good nopl xtopology
nonstop_tsc cpuid aperfmperf pni pclmulqdq dtes64 monitor
ds_cpl vmx est tm2 ssse3 sdbg fma cx16 xtpr pdcm pcid sse4_1
sse4_2 movbe popcnt tsc_deadline_timer aes xsave avx f16c
rdrand lahf_lm abm epb tpr_shadow vnmi flexpriority ept vpid
fsgsbase tsc_adjust bmi1 avx2 smep bmi2 erms invpcid xsaveopt
dtherm ida arat pln pts

\end{lstlisting}


\section{Benchmarks}\label{sec:benchmarks}

\subsection{Memcpy}\label{subsec:memcpy}
    Tested implementations:
    \begin{enumerate}
        \item normal - without fast strings instructions or SSE
        \item with movsb - fast string instruction
        \item with movsq - fast string instruction
        \item with movdqu - SSE instruction
        \item standard C library implementation
    \end{enumerate}





Results - string size: \textbf{8011} , samples: \textbf{100 000}
\begin{lstlisting}[]
MEMCPY
__________|__________|__________|__________|__________
          |       min|       max|       med|rel. stddev
----------|----------|----------|----------|----------
    NORMAL|     11865|     34554|   12315.7|   16.5778
----------|----------|----------|----------|----------
     MOVSB|       102|     10585|     200.2|  111.3528
----------|----------|----------|----------|----------
     MOVSQ|       123|     15054|     212.0|  128.3399
----------|----------|----------|----------|----------
    MOVDQU|       922|     13191|     961.7|   67.3049
----------|----------|----------|----------|----------
  STANDARD|       103|      9367|     201.4|   92.3262
\end{lstlisting}
$NORMAL$ results are obvious. $MOVSQ$ suprisingly is slightly slower than $MOVSB$, these instructions exists mostly for legacy support, so they may be not optimized very well.
$STANDARD$ implementation probably use $MOVSB$, because times minimum (and even medium) times are similar.
Copying 16-byte blocks with $MOVDQU$ is slower, it may be that movsb is well optimized for memcpy task.



\subsection{Strcmp}\label{subsec:strcmp}
    Tested implementations:
    \begin{enumerate}
        \item normal - without fast strings instructions or SSE
        \item cmps
        \item sse42
        \item pcmpeqb - MMX instruction
        \item standard C library implementation
    \end{enumerate}
Results - string size: \textbf{8011} , samples: \textbf{100 000}
\begin{lstlisting}[]
STRCMP
__________|__________|__________|__________|__________
          |       min|       max|       med|rel. stddev
----------|----------|----------|----------|----------
    NORMAL|     14721|     89987|   15435.6|   31.3166
----------|----------|----------|----------|----------
      CMPS|      4232|     22488|    4402.3|   37.6476
----------|----------|----------|----------|----------
     SSE42|       423|     11733|     450.5|  103.8172
----------|----------|----------|----------|----------
   PCMPEQB|       301|     11894|     362.1|  147.4617
----------|----------|----------|----------|----------
  STANDARD|       341|     11321|     368.8|  127.2847
\end{lstlisting}
$STANDARD$ implementation uses $PCMEQB$.
MY $PCMPEQB$ gives slightly better results, maybe binaries linked from standard C library are compiled for other processor model (with different cache, etc.).



\subsection{Strlen}\label{subsec:strlen}
    Tested implementations:
    \begin{enumerate}
        \item normal - without fast strings instructions or SSE
        \item scas
        \item sse42
        \item standard C library implementation
    \end{enumerate}
Results - string size: \textbf{8011} , samples: \textbf{200 000}
\begin{lstlisting}[]
STRLEN
__________|__________|__________|__________|__________
          |       min|       max|       med|rel. stddev
----------|----------|----------|----------|----------
    NORMAL|      3126|     42095|    3323.7|   39.8071
----------|----------|----------|----------|----------
      SCAS|      4754|     37463|    5046.8|   70.1096
----------|----------|----------|----------|----------
       SSE|       164|     11841|     182.0|  134.0813
----------|----------|----------|----------|----------
     SSE42|       468|     10767|     490.9|   81.8084
----------|----------|----------|----------|----------
  STANDARD|       129|     11024|     157.2|  185.7166
\end{lstlisting}

Fast string instruction $SCASB$ is slower than $NORMAL$ implementation in this benchmark.
It may be, because this instruction exists mostly for legacy support.
Similary like in STRCMP results, SSE42 implementation is slower than implementation from standard library.
$STANDARD$ implementation uses MMX instruction pcmpeqb, and SSE pmovmskb and pminub.
My $SSE$ implementation is slower because it doesn't use pminub in optimal way (pminub uses per iteration or sometching like this).


\subsection{Strstr}\label{subsec:strstr}
    Tested implementations:
    \begin{enumerate}
        \item normal - without fast strings instructions or SSE
        \item cmps
        \item sse42
        \item standard C library implementation
    \end{enumerate}
Results - haystack size: \textbf{15556} , needle size: \textbf{25}, samples: \textbf{200 000} \newline
Full needle occurence is at the end of the haystack.
\begin{lstlisting}[]
STRSTR
__________|__________|__________|__________|__________
          |       min|       max|       med|rel. stddev
----------|----------|----------|----------|----------
    NORMAL|     16303|     78276|   16515.1|   30.6925
----------|----------|----------|----------|----------
      CMPS|    265610|    454232|  268041.4|   12.3266
----------|----------|----------|----------|----------
     SSE42|      1528|     14546|    1585.3|   75.4449
----------|----------|----------|----------|----------
  STANDARD|     12343|     69141|   12775.9|   35.1619
\end{lstlisting}
$SSE42$ implementation uses PcmpIstrI instruction with immediate control byte EQUAL ORDERED, which allows to search for needle occurence (not only full, but also needle prefix on haystack suffix) in haystack (botch up to 16 bytes) in constant time, so performance is significantly better.


\end{document}
